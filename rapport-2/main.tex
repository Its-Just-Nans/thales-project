\documentclass{PackagerQualityN}

\title{Projet Tutoré}
\author{Groupe Robot}
\date{Mai 2021}




% TO DO mettre un un footer avec date a gauche et Latex a droite



\begin{document}



\begin{titlepage}%ta page de titre
\newgeometry{left=1cm,right=1cm,top=2cm,bottom=1.5cm}


\includegraphics[height=2.1cm]{./img/uca2.png}
\hfill
\includegraphics[height=2.75cm]{./img/logoThales.png}


\begin{center}
\Huge
Module M2109 - Projet tuteuré\\
Compte-rendu
\\
\vspace{1cm}
\imgNofig{img/robot.jpg}{0.8}
\vspace{1cm}
\large
Membre du groupe:\\
Yahia BEN NAOUA, Maxence CHARLOT, Sêdjro FLASON, Benjamin NEUMANN,\\ Jules SOUBEYRAS et Nans WEBERT\\
\vspace{0.2cm}
Promotion 2019-2021\\
\vspace{1cm}
Sujet de projet : Projet Robot - Groupe 2\\
Tuteurs de projet : M. CAM et Mme ESCAZUT\\
\vspace{1cm}
\includegraphics[height=3cm]{img/logoRetT.png}

\end{center}


\end{titlepage}

%==================================SOMMAIRE========================================

\newp       %nouvelle page custom
\newgeometry{left=2cm,right=2cm,top=2.5cm,bottom=3.5cm}
\vspace*{\stretch{1}}
\setcounter{tocdepth}{2}
\tableofcontents
\vspace*{\stretch{1}}


%==========================================INFO===================================


\vspace*{\stretch{1}} % centrage vertical

%==========================================INFO===================================
\newp
\vspace*{\stretch{1}} % centrage vertical
\section*{Introduction}

Le DUT Réseau et Télécommunications est une formation polyvalente qui nous permet d’acquérir des compétences dans les domaines de l’informatique et des réseaux afin d’être capable de nous adapter aux besoins spécifiques des entreprises à la fin du cursus ou de poursuivre les études en école d’ingénieur ou en licence dans les domaines qui concernent les technologies de l’information et de la communication.\\

Pour mieux appréhender les réalités du travail en entreprise et pour nous donner de l’autonomie dans la réalisation de projets, nous réalisons au cours de notre cursus un projet tuteuré étalé sur trois semestres.\\

Dans le cadre de notre projet tuteuré, notre cheffe de département Mme ESCAZUT a permis a notre groupe exclusivement composé d’alternants la chance de travailler avec une des plus grandes entreprises d’Europe, Thales Alenia Space qui est une entreprise spatiale leader dans la conception de satellite.\\

Le projet que nous a confié M. CAM, Chef du Service Missionisation \& Validation à Thales Alenia Space, consiste à la conception de la troisième version d’un robot autonome pour le compte de l’entreprise. Le robot en est actuellement à sa deuxième version et le but de notre projet est de réaliser le robot autonome version 3 en utilisant des composants « typés spatiaux » afin de permettre de réaliser des prototypes facilement et rapidement.\\

Notre première partie de projet tuteuré consistait à déterminer des solutions pour l’amélioration du robot lors du second semestre. Il s’agissait entre autres de trouver une solution pour les problèmes de tension, le choix des capteurs, de la caméra à intégrer et la refonte du châssis. Au cours des semestres 3 et 4 nous avons revus et fais des modifications sur les solutions choisies au second semestre, et nous les avons mises en application pour la réalisation de la troisième version du robot.\\

Pour ce faire, nous avons étudié les problèmes auxquels nous étions confrontés ainsi que les taches à réaliser et nous avons réparti le travail en sous-groupe.\\

Par la suite, nous vous exposerons les différents problèmes auxquels nous avons été confrontés, les solutions qu’on a appliquées pour les résoudre et les différentes parties de la réalisation du projet en fonctions des solutions proposées pour chaque problème à traiter.\\


\vspace*{\stretch{1}} % centrage vertical
%==========================================INFO===================================


\vspace*{\stretch{1}} % centrage vertical
%==================================TENSION==========================================

\newp       %nouvelle page custom
\paragraph{}

\section{Mise en situation}

Le projet que nous a confié Thales Alenia Space a été réparti en deux groupes différents qui étaient en compétition. Chaque groupe a travaillé sur la mise en place des solutions pour la réalisation de la troisième version du robot.\\

Ces solutions concernaient entre autres tout ce qui touchait à la tension, au rajout de nouveaux capteurs, au rajout d'un nouveau raspberry pour gérer les données que va traiter le robot et à la remodélisation du châssis. Étant donné que les équipements embarqués sur le robot ont des tensions différentes, il fallait  connaître les tensions délivrées par chacun et pouvoir alimenter le robot avec une source adéquate, tout en évitant les problèmes de surchauffe.\\

Le premier groupe avait proposé comme solution l’ajout d’une nouvelle batterie 32 Volts, la déclinaison de la tension électrique en 24 Volts, 12 Volts et 5 Volts, l’utilisation d’un second raspberry, le rajout d’un switch Ethernet, l'ajout de la camera TOF puis enfin d’un nouveau châssis pour accueillir le tout.\\

Le deuxième groupe s’est plus penché sur les problèmes de tension entre le raspberry et l’alimentation et a proposé plusieurs solutions pour régler ce problème. Ensuite, l’utilisation de la camera ToF-Time of Flight étant donné quelle avait déjà été testé et son intégration était possible. Pour finir, il y avait également la refonte du châssis. 

La solution choisie était celle du deuxième groupe, et les deux groupes se sont réunis pour travailler sur le projet pendant le semestre 3 et 4.

\subsection{Exposé des problèmes}

\subsubsection{Le problème de Tension}
Comme énoncé lors du rapport en fin de semestre 2 le robot est constitué de capteurs qui ont besoin d’être alimentés en électricité. Mais malheureusement, le voltage des différents capteurs est différent, soit 24 volts, soit 12 volts, soit 5 volts.

Nous devons donc partir de l’élément primordial qui fournit l’énergie, la batterie, pour ensuite changer le voltage. Notre batterie est composée de 8 piles qui fournissent un total de 32 volts, chaque pile délivre 4 volts.

Il faudra donc gérer toutes les déclinaisons de tension pour pouvoir alimenter comme il faut chacun des équipements présents sur le robot et permettre d'éviter tout ce qui est surchauffé.\\
\\
Plusieurs solutions avaient été proposer notamment le pont diviseur de tension qui utilise les résistances pour abaisser les tensions, le régulateur de tension qui gère automatiquement la tension de sortie en fonction de la tension d’entrée et qui est couplé avec des condensateurs pour éviter des perturbations du signal et pour finir la solution du Buck converter.

\subsubsection{Le schéma 3D}
L’un des objectifs principaux de notre projet tuteuré est de pouvoir embarquer les nouveaux équipements. Pour cela, un problème de place sur le châssis se posait. Il fallait donc l’améliorer et trouver différentes solutions pour que le robot puisse avoir des fondements solides lui permettant de réaliser ce pour quoi il avait été conçu. 
En dehors de la refonte du châssis, plusieurs autres solutions avaient été envisagées. Il s’agissait entre autres de l’utilisation de roues ou de chenilles pour rendre le robot maniable.

Les roues sont moins confortables au niveau de la flottaison, mais consomment moins, elles peuvent porter un peu moins de poids, mais sont idéales sur un sol plat. De plus, elles sont idéales s’il faut une vitesse plus élevée.

Les chenilles quant à elles offrent une flottaison inégalable, mais ont une consommation légèrement plus élevée de carburant (8\%). Les chenilles permettent une charge plus élevée, mais en revanche,le mécanisme d’entraînement des chenilles occasionne plus de résistance, de sorte quelles fournissent moins de puissance à des vitesses élevées.

\subsubsection{Intégration de la caméra}

EXPLICATION DE CE QUE c'est ET COMMENT ça MARCHE 

\img{./img/tof.jpg}{Caméra Time Of Flight}{0.5}

Concernant le capteur caméra ToF Time Of Flight, on avait vu que l’intégration est possible, étant donné que l’étudiant qui avait réalisé la deuxième version du robot l’avait déjà faite.



On en était arriver a la conclusion que c'était la meilleur solution surtout que le caméra ToF était déjà à la disposition de M. CAM.

Néanmoins, les questions sur le traitement des données se posaient toujours. Est-ce qu'il fallait les traiter sur les raspberrys ou sur un ordinateur après l’utilisation d’un raspberry ?   

\newp
\subsection{Le problème de Tension}

Comme énoncé lors du rapport en fin de semestre 2 le robot est constitué de capteurs qui ont besoin d’être alimentés en électricités. Mais malheureusement, le voltage des différents capteurs sont différents, soit 24 volts, soit 12 volts, soit 5 volts.
Nous devons donc partir de l’élément primordial qui fournit l’énergie, la batterie, pour ensuite changer le voltage. Notre batterie est composée de 8 piles qui fournissent un total de 32 volts, chaque pile délivre 4 volts.
Il a fallu gérer toutes les déclinaisons de tension pour pouvoir alimenter comme il faut chacun des équipements présents sur le robot et permettre d'éviter tout ce qui est surchauffé.\\ \\

Plusieurs solutions avaient été proposer notamment le pont diviseur de tension qui utilise les résistances pour abaisser les tensions.

\img{img/PontTension2.png}{Schéma du Pont Diviseur de Tension}{0.7}

Il y avait aussi la solution du régulateur de tension qui gère automatiquement la tension de sortie en fonction de la tension d’entrée et qui est couplé avec des condensateurs pour éviter des perturbations du signal.

\img{img/Regul.png}{Schéma du circuit complet avec les régulateurs et les condensateurs}{0.7}

Pour finir la solution du Buck converter.

\img{img/buck_converter.png}{Schéma du circuit type du Buck Converter}{1.4}

\subsection{Le schéma 3D}

L’un des objectifs principaux de notre projet tuteuré est de pouvoir embarquer les nouveaux équipements. Pour cela, un problème de place sur le châssis se posait. Il fallait donc l’améliorer et trouver différentes solutions pour que le robot puisse avoir des fondements solides lui permettant de réaliser ce pour quoi il avait été conçu, c'est-à-dire pouvoir réaliser des prototypes facilement et rapidement.
En dehors de la refonte du châssis, plusieurs autres solutions avaient été envisagées. Il s’agissait entre autres de l’utilisation de roues ou de chenilles pour rendre le robot maniable.

\img{img/chenil1.png}{Modèle de chenille}{1.5}

\img{img/roue.JPG}{Modèle de roue}{0.6}

Les roues sont moins confortables au niveau de la flottaison, mais consomment moins, elles peuvent porter un peu moins de poids, mais sont idéales sur un sol plat. De plus, elles sont idéales s’il faut une vitesse plus élevée.\\

Les chenilles quant à elles offrent une flottaison inégalable, mais ont une consommation légèrement plus élevée de carburant (8\%). Les chenilles permettent une charge plus élevée, mais en revanche,le mécanisme d’entraînement des chenilles occasionne plus de résistance, de sorte qu’elles fournissent moins de puissance à des vitesses élevées.

\subsection{Intégration de la caméra}
 
Une caméra ToF - Time of Flight encore appelé caméra Temps de vol  est une caméra qui permet en plus de capturée des images d'avoir une information de profondeurs pour chaque pixel des images qui ont été capturées. \\

\img{./img/tof.jpg}{Caméra Time Of Flight}{0.5}

Pour évaluer une distance, la caméra envoie une impulsion de lumière proche de l’infrarouge à l’aide de plusieurs LEDs autour de l’objectif. Ce système est un peu basé sur la façon dont une chauve souris perçoit son environnement. Cette impulsion émise atteint son sujet et revient au capteur. Le capteur vient calculer par la suite le temps mis par l’impulsion à revenir jusqu’aux capteurs fournissant ainsi des données cartographiques de la profondeur. Cette mesure de temps de vol est effectuée indépendamment pour chaque pixel de la caméra, permettant ainsi d’obtenir une image complète en 3D de la scène filmée.\\

\img{./img/infrarougeC.jpg}{Fonctionnement de la lumière infrarouge}{0.5}

Pour avoir une mesure précise d’un objet, il faut qu’un seul rayon lumineux revienne sur l’objectif. Dans le cas contraire, la caméra va mal interpréter le signal reçu et perdre en précision. 
Il faut également éviter les miroirs ou les objets trop réfléchissants, car ils dévient la quasi-totalité de l’impulsion envoyée par la caméra et elle ne reçoit donc plus le signal en retour permettant de détecter l’objet en question.\\


La caméra ToF offre un énorme avantage par rapport aux autres technologies, car elle permet d'effectuer la  mesure avec précision des distances dans une scène complète avec une unique impulsion laser.\\

Pour résumer, la caméra ToF:
\begin{itemize}
    \item Émet un signal lumineux infrarouge
    \item Mesure le temps d'aller-retour du signal
    \item Détermine la profondeur en fonction des données extraites\\
\end{itemize}

La caméra communique par Ethernet grâce au protocole GigE Vision, un protocole homogène basé sur UDP/IP.
La technologie de caméra à temps de vol (ToF) à plusieurs utilisations. On peut s'en servir dans des smartphones par exemple. Pour notre projet, on l'utilisera pour effectuer des captures d'images et de vidéo sur le robot. Il est utilisé entre autres pour le balayage d'objets, la navigation intérieure, la reconnaissance gestuelle l'imagerie 3D et l'amélioration des expériences de réalité augmentée.\\

La caméra fournit 3 images différentes des scènes qui sont filmées, chaque image est codée avec des pixels de 16 bits en nuances de gris.
Nous avons donc une image qui montre la luminosité reçue par chaque pixel suite à l’impulsion. Plus un objet est proche, plus les pixels qui le composent sont clairs.
Ensuite une image qui montre la fiabilité de la mesure pour chaque pixel. Plus un pixel est clair, plus sa mesure est fiable.
Pour finir une image qui représente la distance de chaque pixel par rapport à l’objectif. Plus un pixel est clair, plus il est proche de l'objectif de la caméra.\\
À partir de ces 3 images, la caméra crée automatiquement un nuage de points en 3D avec pour chaque pixel leurs coordonnées XYZ dans l’espace en millimètres.
La conversion de la valeur d’un pixel  se fait par calcul en connaissant les réglages de distance minimum et maximum de la caméra.
Pour intégrer la caméra avec ROS, le package qui a été utilisé est basler\_tof utilisant la librairie GenTL pour communiquer avec la caméra.\\

\img{./img/ezR.jpg}{Images capturées avec la caméra ToF}{0.5}

Il est possible d’envoyer des fichiers ou des commandes au Raspberry depuis un autre ordinateur sur le même réseau, cela évite de devoir brancher d'autres périphériques comme un clavier , une souris et un écran pour chaque manipulation laissant le robot libre de se déplacer sans câbles derrière lui. Pour cela, on utilise le protocole SSH qui sécurise une connexion entre deux périphériques pour l’échange de commandes. Il existe différents logiciels de client SSH en fonction du système d’exploitation utilisé ainsi que des fonctionnalités souhaitées.\\

Concernant le capteur caméra ToF Time Of Flight, on avait vu que l’intégration est possible, étant donné que l’étudiant qui avait réalisé la deuxième version du robot l’avait déjà faite.
On en était arrivé à la conclusion que c'était la meilleure solution surtout que la caméra ToF était déjà à la disposition de M. CAM.
Néanmoins, les questions sur le traitement des données se posaient toujours. Est-ce qu'il fallait les traiter sur les Raspberrys ou sur un ordinateur après l’utilisation d’un Raspberry ?  

\subsection{Prévision des tâches}

\img{./img/prevision1.JPG}{Diagramme de Grantt des tâches du S3 et du S4}{0.4}

Les prévisions qu’on avait faites au début du semestre 3 ont connu de nombreux changements au retard du matériel qu’on aura dû recevoir. 
Les tests pour le buck converter ont pu être réalisés en partis, mais il a fallu les mettre sur pause étant donné qu’on avait besoin de la caméra pour pouvoir effectuer la suite.\\
L’intégration de la caméra n’a donc pas pu débuter au moment opportun. Les tâches effectuées en attendant sa disponibilité étaient la prise en main du robot après la démonstration de M. CAM.
Il a fallu ensuite comprendre le fonctionnement de ROS, comment il fallait l’installer puis comprendre les packages qu’il faudrait utiliser ainsi que le code produit par l’alternant ayant travaillé avec M. CAM.\\
Le groupe chargé de la refonte du châssis a pu commencer à regarder quels logiciels étaient mieux adaptés pour réaliser la modélisation. Après avoir testé et travaillé sur plusieurs logiciels le choix c’est porté sur Fusion 360 et le groupe a commencé à le prendre en main.\\

Étant donné que le câble qu’on aurait dû utilisé pour connecter le robot était parti en Islande à cause d’un souci logistique, nous n’avons reçu la caméra qu’en mars ce qui nous a ralentis dans la bonne exécution du diagramme de Gantt qu’on avait réalisé.\\

Dès sa réception, le groupe de modélisation a réussi à prendre les mesures de la caméra pour pouvoir retravailler sur le support de caméra et poursuivre les autres tâches à savoir apporter les modifications effectuées par M. CAM et Mme ESCAZUT lors de la précédente réunion ainsi que l'étude des forces supportées par le châssis.\\ 

Après s’être concentré sur le fonctionnement de ROS et sur le code de l’alternant avec lequel avait travaillé M.CAM, le groupe responsable de l’intégration de la caméra a réussi à la faire fonctionner en réalisant quelques prises de vue.\\

Il nous manquait quand même quelques pièces pour avancer dans la modélisation et dans les autres tâches comme le support pour les piles, les roues et le moteur pour le robot. Nous avons pu récupérer des supports des piles au secrétariat de l’IUT le 4 juin et les roues le 11 juin lors de la dernière réunion avec M. CAM et Mme ESCAZUT. 
\img{./img/grantT.jpg}{Diagramme de Grantt des tâches du S3 et du S4 2.0}{0.4}

\section{Recherche de solutions}

\subsection{Recherche en groupe}


Deux groupes étaient en charge du projet lors du deuxième semestre. Ces deux groupes étaient en compétition. L'objectif était, pour chacun des deux groupes, de trouver des solutions pour améliorer le robot afin qu'au début du troisième semestre, M.CAM et Mme ESCAZUT choisissent les meilleures solutions.\\

Les solutions qui ont finalement étaient retenue étaient celles proposés par le groupe composé de Maxence CHARLOT, Jules SOUBEYRAS et Nans WEBERT. Le second groupe composée de Sedjro FLASON, Benjamin NEUMANN ainsi que Yahia BEN NAOUA ont donc rejoins le groupe avec les solutions les plus intéressantes pour travailler à 6 sur le projet.\\

Le groupe a donc revu ensemble l'ensemble des problèmes à résoudre sur le robot, réfléchis sur les solutions à mettre en place pour les résoudre, évaluer le temps nécessaire pour la mise en oeuvre du projet et fait la liste des matériaux nécessaires pour réaliser le projet.


\subsection{Solutions retenus}

\subsubsection{Le problème de Tension}

La solution retenue pour régler ce problème est le Buck Converter. Le Buck Converter permet entre autres d’obtenir une tension réglable, mais toujours inférieure à celle présente à l’entrée. Il a donc un rôle de variateur de tension continue. C’est donc précisément notre cas.

Cette solution a donc pour avantage de n’émettre presque pas de chaleur grâce à l’inducteur, car un inducteur idéal ne dissipe pas l’énergie sous forme de chaleur.

\subsubsection{Le schéma 3D}

L’une des solutions retenues à la fin du semestre 2  était les chenilles pour des raisons de maniabilité. 
Néanmoins au début du semestre 3 M.CAM a décidé qu’on partirait plutôt sur des roues, mais des roues adaptées au terrain que devrait emprunter le robot.

On est également parti sur une nouvelle modélisation totale du châssis en prenant en compte les nouveaux équipements qu’il doit embarquer. Il s’agit donc de deux Raspberrys, d’un switch, d’un support pour la caméra ToF, de l’alimentation et du moteur. 

\subsubsection{Intégration de la caméra}

M.CAM avait comme souhait que le robot soit stable le plus longtemps possible. L'ancien étudiant avait travaillait sur ce projet avec une distribution de ROS nommée Kinetic qui avait son EOL (end-of-life) en avril 2021. M.CAM nous a donc demandé de prendre la distribution la plus récente de ROS afin de pouvoir travaillé dans le temps sur ce projet sans éventuels soucis.

Cette dernière distribution est ROS Noetic Ninjemys. C'est une version LTS (long-time-support). Ce qui veut dire que cette version est faite pour avoir une durée de vie plus longue que la normale.\\

\img{./img/ROS_Version_distro.png}{Liste des distributions de ROS actuelle avec leur date de parution, leur logo et leur date de fin de vie}{0.6}

Le fait qu'ai utilisé la toute dernière distribution de ROS permettras aux personnes voulant reprendre le robot de pouvoir utiliser exactement les mêmes lignes de code que nous. Cette distribution a une EOL (end-of-life) prévue pour mai 2025. Après cette date il faudra probablement utiliser une nouvelle distribution car celle-ci sera obsolète. Notre code ne sera donc plus utilisable de la même manière que nous l'avons fait.




\newp
\section{Réalisation}

Pour réaliser les différentes tâches, nous nous sommes repartis en plusieurs sous-groupes en fonction des problèmes qu'on avait à résoudre.

La première organisation était composée de quatre groupes:
\begin{itemize}
\item Le groupe du problème de tension formé par Maxence CHARLOT
\item Le groupe de la modélisation formé par Yahia BEN NAOUA et Benjamin NEUMANN
\item Le groupe raspberry et caméra formé par Nans WEBERT
\item Le groupe de soutien formé par Sedjro FLASON et Jules SOUBEYRAS qui était aussi chargé de faire le compte rendu
\end{itemize}

Ensuite, la solution du Buck Converter ayant été rapidement testé on est passé a une réorganisation des groupes:
\begin{itemize}
\item Le groupe de la modélisation maintenant formé par Yahia BEN NAOUA, Sedjro FLASON et Benjamin Neumann
\item Le groupe raspberry et caméra formé par Maxence CHARLOT, Jules SOUBEYRAS et Nans WEBERT
\item Le groupe formé par Sedjro FLASON et Jules SOUBEYRAS est resté inchangé pour la rédaction du compte rendu
\end{itemize}

Nous vous expliquerons donc plus loin le travail effectué par chaque groupe, les difficultés auxquelles chacun d'eux a fait face et comment elles ont été surmontées.

\subsection{Groupe modelisation}

Le nouveau robot que notre groupe à réaliser embarque plus d'équipement que le précédent. Pour lui permettre de supporter la charge de ces équipements et de pouvoir être maniable, il fallait repenser le châssis du robot. Un modèle de ce nouveau châssis nous a été proposé par M. CAM en début de projet mais il ne semblait pas vraiment solide.

\img{img/robotProotypeV3.png}{Prototype du robot}{0.6}

Pour cela, nous avons dû modéliser un nouveau châssis nous basant sur celui qui existe et en regardant où et comment est ce qu'il fallait placer les nouveaux équipements.
Le premier point de cette étape était de choisir le bon logiciel de modélisation. On a eu le choix entre plusieurs

\img{img/modélisationLogiciel.jpg}{Logicels de modélisation}{2}

Pour réaliser le modèle 3D du nouveau châssis, nous avions besoin d'un logiciel gratuit et avec lequel nous pourrions facilement collaborer au sein de l'équipe. En plus de ça, il nous fallait également un logiciel qu'on pouvait prendre en main sans trop de difficulté.
Notre choix s’est donc porté sur Fusion 360.\\

Il s'agit d'un logiciel gratuit pour une utilisation personnelle, mais payante pour une utilisation collaborative. Néanmoins, nous avions la possibilité de sauvegarder le modèle conçu et de nous le partager dans un espace de stockage en ligne (Google drive).\\
\\
Nous avons commencé à réaliser un prototype, une première version qui ne permettait de pouvoir bien prendre en main du logiciel et également de pouvoir faire des estimations en attendant d'avoir plus de détails sur certains équipements. Il s'agissait notamment de savoir si le robot embarquerait un seul raspberry ou deux raspberrys mis en cluster puis d'avoir la caméra ToF pour évaluer les dimensions de cette dernière.

\img{img/robotPremierTest.JPG}{Premier test du robot}{0.6}

Avec les autres membres du groupe, nous avons regarder par la suite quel pouvait être le meilleur moyen de répartir les équipement sur le chassis afin que le robot ne soit pas soumis a des charges disproportionnées par endroit et qu'au final il ne soit pas maniable. Nous avons donc convenu du modèle ci-dessous.

\img{img/dessinModèleV3.png}{Dessin du robot}{0.3}


Après avoir finalement reçu la caméra, nous avons pu avoir accès à sa fiche technique qui renseignait toutes ces dimensions. Les conclusions du groupe chargé de travailler sur le raspberry et l'intégration de la caméra ont amené à l'utilisation de deux raspberrys. Ces changements correspondaient bien au schéma réaliser ci-dessus. Nous avons donc pu les appliqué au robot tout en prévoyant un moyen de pouvoir fixer les équipements sur le châssis. Pour cela nous avons réalisé une modélisation avec les équipements et une version imprimable.


\img{img/robotImprimable.JPG}{Modélisation sur fusion360}{0.4}


\subsection{Groupe Tension}

%- La premiere étape était de se pencher sur la partie pratique du problème de tension. Le groupe chargé de cette tâche à donc réussi à manipuler le Buck converter et à transformer une tension d'entrée élevé et la réguler à la baisse en sortie tout en étant parfaitement stable. 
Le problème concernant la tension est une tâche sur laquelle on s’est penché directement. En effet dès lors que l’on nous a parler de ce projet tuteuré lors de la première intervention de M. CAM, ce souci concernant les multiples tensions différentes qui étais nécessaire au bon fonctionnement de chaque équipement du robot à était évoqué. L’utilisation d’un buck converter est la solution finale que nous avons décidé de retenir étant peu couteuse, efficace et avec très peu de perte. 
Voici à quoi ressemble le schéma du circuit de base d’un buck converter

\img{img/buck_converter.png}{Schèma du circuit de base d'un buck converter}{0.9}


Le second problème que le groupe à rencontrer était celles des câbles. En effet on n’avait pas les câbles pour brancher la caméra. Mais après un moment d'attente  M. CAM à finalement réussis a avoir le câble et nous la apporter. 
\\
\img{img/cableCam.png}{Câble pour la caméra}{0.9}

Afin de tester notre option du buck converter nous avons demandé une salle de TP d’électronique à l’IUT. Une fois dans notre salle de TP, nous avons pris notre "breadboard" qui est est une petite plaque en plastique parsemée de trous dans lesquels nous pouvons brancher plusieurs composants électroniques, nos câbles afin de pouvoir se brancher, des pinces crocodiles et notre buck converter. En mettant une tension à l’entrée du buck converter, on voulait que même si la tension d’entrée variait la tension en sortie du buck converter, réguler à la baisse, devais rester parfaitement stable. Après branchement, on avait réussi. On avait trouvé la solution peu couteuse et efficace qui marchait parfaitement.

\img{img/buckTp.jpg}{Test du buck converter en salle de TP}{0.3}



%- explication les câbles et probleme de tension
%- expliquer le problème
%- expliquer la résolution (salle de TP, test etc..) par @MAXENCE CHARLOT 
%- photo à mettre


\subsection{Groupe Raspberry et camera}

Le groupe chargé de cette partie avait pour défi premier la manipulation du robot V2. Il s'est donc penché sur la manipulation du robot, sur les futures tâches qu'il va pouvoir accomplir et a pu initialiser la première connexion avec le robot et à le faire rouler.
Ils ont commencé à regarder les différents fichiers nécessaires à son bon fonctionnement et à réfléchir à ce que l'on pourrait y apporter.\\

\img{img/robotrobot.jpg}{Test pour faire fonctionner le robot}{0.1}

Un des points importants était de comprendre le fonctionnement de ROS pour pouvoir s'en servir dans la conception du robot v3.

\subsubsection{Explication de ROS}

ROS (Robot Operating System) est un système d’exploitation utilisé dans le développement des logiciels pour la robotique tout comme les systèmes d’exploitation pour ordinateur, serveur, etc.
Avant la conception de ROS les concepteurs de robot devaient non seulement concevoir la partie matérielle de leur robot, mais également la partie logiciel associée. ROS vient proposer des fonctionnalités standard à la robotique afin d’éviter aux concepteurs de devoir à chaque fois recréer de nouveaux systèmes.\\

ROS est basé sur 5 grands principes à savoir :
\begin{itemize}
\item 	Peer to peer : qui est un système qui permet les échanges entre les acteurs connectés au système sans transiter par un serveur central. Ici, chaque acteur joue le rôle de clients et de serveurs.
\item 	Basé sur des outils : ROS est basé sur un système "microkernel" qui ne contient que le code de base pour permettre la communication entre l’OS et la partie matérielle. Chaque commande de ROS est en fait un exécutable formant de nombreux petits outils permettant de faire tourner le code.
\item 	Multi langage : il n’y a pas de langage spécifique pour programmer avec ROS
\item 	Léger : pour pallier la difficulté de réutilisation des algorithmes de développement qui pourraient être liés a un quelconque OS de robotique, les pilotes et algorithmes de ROS sont des fichiers exécutables indépendants réutilisables plusieurs fois ce qui permet de maintenir la taille réduite de ROS.
\item 	Gratuit et open source
\end{itemize}

Plusieurs concepts sont utilisés par ROS pour son fonctionnement. Il s'agit principalement des nœuds (Nodes), des topics, des messages, des services. \\

Un noeud est une instance d'un exécutable. Il peut par exemple correspondre à un capteur ou à un moteur présent sur le robot.\\

L'échange d'informations entre les noeuds se font de manière asynchrone par un topic ou de manière synchrone par un service.\\

Un topic est un système de transport de données basées sur les concepts de "subscribe" (abonnement) et de "publish" (publication).
Plusieurs noeuds pourront donc publier et lire des informations sur un topic. Le topic est dit "typé" parce que les messages sont toujours structurés de la même manière.\\

Un service par contre est une notion de communication entre deux noeuds basés sur le système de requête et de réponse.\\

Chaque noeud qui se lance se déclare au master.
Le master est un service d'enregistrement et de déclaration des noeuds qui permet aux noeuds de se connaître et d'échanger des informations. Il comprend une sous partie appelé "parameter server" qui est comme une base de données centralisée ou les noeuds peuvent sauvegarder des informations.\\

\img{img/rosR.jpg}{Schéma de ROS}{0.7}


Comme dit plus tôt, ROS ne dépend d’aucun langage spécifique de programmation. Trois librairies sont définies principalement pour programmer en ROS à savoir, Python, C++ et LISP.

Pour finir, les fichiers de ROS sont organisés  de façon hiérarchique en package et en stack. Un package est un répertoire qui contient les nœuds, les librairies externes, les fichiers de configuration. Un stack est un ensemble de packages.

%- explication raspberry et caméra (faire le début, le groupe s'occupera du reste)
%- explication de ROS (voir cette viédo et la cité https://www.youtube.com/watch?v=iLiI_IRedhI )
%- explication technique de l'API de la camera @méchant nans 
%- explication des solutions mises en places @méchant nans

\subsection{Groupe Compte-rendu}

Notre groupe de compte-rendu avait pour tâche initiale de rédiger les bilans des réunions auxquelles on a participé puis de rédiger également le rapport du tuteuré. Les réunions étaient faites toutes les deux semaines. Au cours de ces réunions, nous devions montrer notre avancement sur le projet. Ces réunions se sont passées en téléconférence par téléphone et sur Microsoft Teams, mais aussi en présentiel à l’IUT lorsque M. CAM était disponible.\\

Entre les membres du groupe, plusieurs réunions ont également été faites pour faire le point sur l'avancement de chaque groupe. Le compte rendu de quelques-unes de ces réunions a été fait en annexe.

Nous avons décidé en début de projet que notre groupe se scinderait en deux. Sedjro irait apporter de l’aide au groupe de modélisation, mais aussi vérifier que le groupe avance bien et dans les temps. Jules lui irait avec Maxence pour les problèmes de tension puis ils fusionneraient par la suite avec Nans pour le travail sur les Raspberry,  sur ROS, et sur la caméra.\\

Étant au centre des activités de chaque groupe cela nous a permis de les aider et pouvoir transcrire le travail qui a été réalisé. Nous avons donc pu à partir de là rédiger le rapport du projet.





\newp
\section{Conclusion}

\subsection{Version finale du projet}

\subsection{apport de ce projet dans notre vie pro}


\subsection{apport de ce projet dans nos compétences}
certain membre ont du apprendre le \LaTeX

\newp  

\appendix
\section*{Annexe - Compte rendu de la réunion du 14 Novembre 2020}
La première réunion que nous avons faite a eu lieu le 14 Novembre 2020. Elle s'est réalisée avec tout les membres de notre groupe.
\\

Juste avant cette réunion les deux groupes en charge du projet robot ont fusionnés. La première réunion à donc permis au groupe dont le projet a été retenu de faire le point au nouveau membre. 
\\

Les nouveaux membres ont donc reçu le dossier de cadrage, puis ils ont dû prendre connaissance des tâches à effectuer. Les problèmes à résoudre ont été redéfini et le travail à donc été réparti en 4 groupe. Un groupe (Maxence) se charge de régler le problème de tension, un autre (Nans) se charge de trouver comment intégrer la nouvelle caméra, le troisième (Yahia et Benjamin) se charge de faire la modélisation 3D du robot et le dernier groupe (Sedjro et Jules) se charge de faire le lien entre les différents groupe pour s'assurer que le travail avance bien et également de rédigé le compte-rendu.
\\
\\
Le point à l'issue de cette réunion a été de partir sur:\\
\begin{itemize}
\item  un robot avec deux caméras (TOF et BlackFly)\\
\item   un robot avec des chenilles\\
\item  les batteries a porter par le robot\\
\item   un switch\\
\item   un raspberry 3 (on verra pour upgrade avec deux ou 1 raspberry + un compute module) donc il fallait prévoir une potentielle mise à jour dans la modélisation.
\end{itemize}

\newp
\section*{Annexe - Compte rendu de la réunion du 21 Novembre 2020}
La deuxième réunion que nous avons faite a eu lieu le 21 Novembre 2020. Elle s'est réalisée avec tout les membres de notre groupe.\\

Pendant cette réunion, nous avons travaillé sur une présentation du travail à réaliser avec les groupes qui ont été définis.\\

Trois problèmes ont été présenté, le problème de tension, la réalisation du schéma 3D et l'intégration de la caméra.
En effet, les composants du robot ont besoin d’être alimentés avec plusieurs tensions différentes et ce problème sera traité en utilisant la solution du Buck converter.
\\
Ensuite, il va falloir ajouter de nouveaux éléments au robot, il faut donc revoir toute la conception du châssis.
Les membres du groupe chargé de cette tâche vont alors s’inspirer de l’ancien visuel pour développer le nouveau, en prenant compte des contraintes physiques sur les roues, par exemple, il faudra utiliser un coefficient de sécurité.
La solution des chenilles a donc été écarté.
\\
Pour finir, nous devons ajouter une caméra au robot, nous devons donc l’ajouter au système ROS pour pouvoir l’utiliser et en tirer des informations.
\\

Nous avons par la suite effectué un plan de mise des oeuvres des différentes tâches pour les différents groupes.
Pour le problème de tension, la première étape était de mettre en place le circuit après réception des composants puis réaliser des tests pour les tensions et courant disponible.
\\
Pour la conception du schéma 3D, la première étape était de créer un nouveau modèle 3D, sur un logiciel de modélisation 3D, comme blender ou un autre logiciel imposé, ensuite l'impression du châssis et pour finir l'ajout de nouveaux éléments.
\\
En ce qui concerne l'ajout de la caméra, la première partie consiste à la mise en place de la configuration du robot et voir comment elle marche, ensuite faire le développement pour l'intégration de la caméra et pour finir faire les tests sur le robot.
\\
Pour finir nous avons fait le point sur les matériaux nécessaires pour la réalisation des différentes tâches.

\newp
\section*{Annexe - Compte rendu de la réunion du 27 Novembre 2020}
La 3ème réunion que nous avons faites a eu lieu le 27 Novembre 2020. Elle s'est réalisée avec tout les membres de notre groupe ainsi qu'avec M. CAM et Mme ESCAZUT.

Au cours de cette rencontre qui à eu lieu a l'IUT, nous avons présenter le travail qu'on aura à réaliser les semaine qui suivront, les problèmes du robot et les solutions pour les résoudre et l'organisation de notre groupe pour pouvoir résoudre ces problèmes.
\\

Nous avons aussi présenté la liste des matériaux dont nous aurons besoin.
\\

M. CAM et Mme ESCAZUT nous ont demandé de faire une liste plus exhaustive des matériaux et ont fait le point sur les matériaux que l'IUT pouvait nous fournir et sur ceux que M. CAM devait nous fournir.


\newp
\section*{Annexe - Compte rendu de la réunion du 6 Janvier 2021}
La 4ème réunion que nous avons faites a eu lieu le 6 Janvier 2021. Elle s'est réalisée avec tout les membres de notre groupe.

Pour présenter le département R\&T et la formation au futurs étudiants, l'université organise des journées portes ouvertes où les projets tuteurés sont mis en avant. Compte tenue de la situation sanitaire actuelle, il fallu organisé cette journée porte ouverte virtuellement.

On a donc été chargé de réaliser une vidéo où l'on expliquait notre projet tuteuré, en quoi il consistait, et qu'elles sont les travaux qui avait été réalisé jusque là.

Notre réunion a donc consisté à faire les points sur les différents thèmes à aborder et comment il fallait réaliser la vidéo qui devait résumé tout le travail du dernier semestre et les différentes tâches à accomplir pour la suite.


\newp
\section*{Annexe - Compte rendu de la réunion du 14 Janvier 2021}
La 5ème réunion que nous avons faites a eu lieu le 14 Janvier 2021. Elle s'est réalisée avec tout les membres de notre groupe ainsi que M. CAM.


Pendant cette réunion M. CAM nous a enfin confié le robot. Il en a profité pour nous faire un test du fonctionnement du robot. Nous avons ainsi pu voir comment il marchait, quels étaient les principaux fichiers de configuration, comment fonctionnait la caméra déjà intégré au robot puis nous a expliqué à nouveau le processus de tracking du robot.
\\

Il nous a également fourni les matériaux dont ont avons besoin pour monter le robot.
\\

Nous en avons profiter pour faire quelques plans à rajouter à notre vidéo.

\newp       %nouvelle page custom

\listoffigures

\newp

\bibliographystyle{unsrt}
\bibliography{references}



\end{document}