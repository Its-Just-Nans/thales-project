\subsection{Prévision des tâches}


Insérer image de grann
dire que tout c'eest pas passé comme prévu

ecpliquer le cable qui est parti en islande


le retard de la caméra



\section{Recherche de solutions}

\subsection{Recherche en groupe}


expliqué les deux premiers semestre, qu'on était en concurence, les solutions des deux groupes etc...

\subsection{Solutions retenus}

\subsubsection{Le problème de Tension}

La solution retenue pour régler ce problème est le Buck Converter. Le Buck Converter permet entre autres d’obtenir une tension réglable, mais toujours inférieure à celle présente à l’entrée. Il a donc un rôle de variateur de tension continue. C’est donc précisément notre cas.

Cette solution a donc pour avantage de n’émettre presque pas de chaleur grâce à l’inducteur, car un inducteur idéal ne dissipe pas l’énergie sous forme de chaleur.

\subsubsection{Le schéma 3D}

L’une des solutions retenues à la fin du semestre 2  était les chenilles pour des raisons de maniabilité. 
Néanmoins au début du semestre 3 M.CAM a décidé qu’on partirait plutôt sur des roues, mais des roues adaptées au terrain que devrait emprunter le robot.

On est également parti sur une remodélisation totale du châssis en prenant en compte les nouveaux équipements qu’il doit embarquer. Il s’agit donc de deux raspberrys, d’un switch, d’un support pour la caméra ToF, de l’alimentation et du moteur. 

\subsubsection{Intégration de la caméra}

dire qu'il fallait intégrer la caméra qsdfdsqfsqfsdqfqsdfqsdf
